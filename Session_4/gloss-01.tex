%!TEX TS-program = Arara
% arara: pdflatex: {shell: yes}
% arara: makeglossaries
% arara: pdflatex: {shell: yes}

\documentclass[12pt,ngerman]{scrartcl}
\usepackage{babel}
\usepackage[acronym]{glossaries}
\makeglossaries

\newglossaryentry{latex}
{
        name=latex,
        description={eine Markup-Sprache für wissenschaftliche Text}
}
\newacronym{kgv}{KGV}{Kleines Gemeinsames Vielfaches}

\newacronym{ggt}{GGT}{Größter Gemeinsamer Teiler}

\begin{document}

Die \Gls{latex} Markup-Sprache ist nützlich in den Wissenschaften, aber nicht nur dort.

\textbf{acr-Befehle}

\acrlong{kgv}

\acrshort{kgv}

\acrfull{kgv}

\textbf{gls-Befehle}


\gls{kgv}

\gls{ggt}

\clearpage

\gls{kgv}

\gls{ggt}



\printglossary[type=\acronymtype]

\printglossary

\end{document}