\documentclass[12pt,ngerman,parskip=half]{scrreprt}
\usepackage{blindtext}

\usepackage[T1]{fontenc}
\usepackage{xcolor}
\usepackage{graphicx}
\usepackage{booktabs}
\usepackage{paralist}
\usepackage{listings}
\usepackage{csquotes}
\usepackage{lmodern}

\usepackage{babel}
\usepackage{microtype}

\author{Uwe Ziegenhagen}
\title{Mein erstes \LaTeX-Dokument}

\newcommand{\person}[1]{\textsc{#1}}

\newcommand{\pperson}[2]{\textbf{#1}~\textsc{#2}}

\begin{document}
\maketitle

\tableofcontents

\chapter{Themenüberblick}

\section{Einführung}
\subsection{Literaturüberblick}
\subsubsection{Literatur vor 1900}

Hallo, ich {\tiny bin} ein {\Huge großer} Text. \person{Albert Einstein} \pperson{Albert}{Einstein} 

Siehe dazu auch Abschnitt \ref{sec:hauptteil} auf Seite \pageref{sec:hauptteil}.

\person{Albert Einstein}\footnote{Bekannter Physiker}\marginpar{Albert Einstein war ein bekannter Physiker, der sich mit der Relativitätstheorie beschäftigte.}

Albert Einstein sagte einmal: \enquote{Glaubt nicht alles, \enquote{Glaubt nicht alles, was ihm im Internet lest!} was ihm im Internet lest!}

Albert Einstein sagte einmal: \glqq Glaubt nicht alles, was ihm im Internet lest!\grqq


\textmd{Uwe Ziegenhagen}

\textbf{Uwe Ziegenhagen}


Hallo, ich bin \textit{Text} \textit{\textbf{Text}} \textbf{\textit{Text}}   ein \texttt{Typewriter} \textbf{fettgedruckter} Text

\blindtext

\blindtext

\blindtext

\subsubsection{Literatur nach 1900}

\blindtext

\blindtext

\blindtext

\section{Hauptteil}\label{sec:hauptteil}

\blindtext[100]


\section{Schlussworte}

\blindtext[10]


\end{document}

