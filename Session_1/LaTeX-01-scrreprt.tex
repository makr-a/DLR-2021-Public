\documentclass[12pt,ngerman,parskip=half]{scrreprt}
\usepackage{blindtext}

\usepackage[T1]{fontenc}
\usepackage{xcolor}
\usepackage{graphicx}
\usepackage{booktabs}
\usepackage{paralist}
\usepackage{listings}
\usepackage{csquotes}
\usepackage{lmodern}

\usepackage{babel}
\usepackage{microtype}

\author{Uwe Ziegenhagen}
\title{Mein erstes \LaTeX-Dokument}

\newcommand{\person}[1]{\textcolor{red}{\textsc{#1}}}

\newcommand{\pperson}[2]{\textbf{#1}~\textsc{#2}}

\usepackage{hyperref}
\hypersetup{
    bookmarks=true,                     % show bookmarks bar
    unicode=false,                      % non - Latin characters in Acrobat’s bookmarks
    pdftoolbar=true,                        % show Acrobat’s toolbar
    pdfmenubar=true,                        % show Acrobat’s menu
    pdffitwindow=false,                 % window fit to page when opened
    pdfstartview={FitH},                    % fits the width of the page to the window
    pdftitle={My title},                        % title
    pdfauthor={Author},                 % author
    pdfsubject={Subject},                   % subject of the document
    pdfcreator={Creator},                   % creator of the document
    pdfproducer={Producer},             % producer of the document
    pdfkeywords={keyword1, key2, key3},   % list of keywords
    pdfnewwindow=true,                  % links in new window
    colorlinks=true,                        % false: boxed links; true: colored links
    linkcolor=blue,                          % color of internal links
    filecolor=blue,                     % color of file links
    citecolor=blue,                     % color of file links
    urlcolor=blue                        % color of external links
}

\begin{document}
\maketitle

\tableofcontents

\chapter{Themenüberblick}

\section{Einführung}
\subsection{Literaturüberblick}
\subsubsection{Literatur vor 1900}

Hallo, ich {\tiny bin} ein {\Huge großer} Text. \person{Albert Einstein} \pperson{Albert}{Einstein} 

Siehe dazu auch Abschnitt \ref{sec:hauptteil} auf Seite \pageref{sec:hauptteil}.

\person{Albert Einstein}\footnote{Bekannter Physiker}\marginpar{Albert Einstein war ein bekannter Physiker, der sich mit der Relativitätstheorie beschäftigte.}

Albert Einstein sagte einmal: \enquote{Glaubt nicht alles, \enquote{Glaubt nicht alles, was ihm im Internet lest!} was ihm im Internet lest!}

Albert Einstein sagte einmal: \glqq Glaubt nicht alles, was ihm im Internet lest!\grqq


\textmd{Uwe Ziegenhagen}

\textbf{Uwe Ziegenhagen}

\textcolor{red}{Ich bin roter Text}

Hallo, ich bin \textit{Text} \textit{\textbf{Text}} \textbf{\textit{Text}}   ein \texttt{Typewriter} \textbf{fettgedruckter} Text

\blindtext \clearpage

\begin{itemize}
	\item Hallo
	\item ich bin
	\item eine 
	\item itemize
	\item Umgebung
	\item in LaTeX
\end{itemize}

\begin{compactitem}[\textcolor{red}{$\Rightarrow$}]
	\item Hallo
	\item ich bin
	\item eine 
	\item itemize
	\item Umgebung
	\item in LaTeX
\end{compactitem}

\begin{enumerate}
	\item Hallo
	\item ich bin
	\item eine 
	\item itemize 
	\item Umgebung
	\item in LaTeX
\end{enumerate}

\begin{compactenum}[1.] \setcounter{enumi}{2} 
	\item Hallo
	\item ich bin
	\item eine 
	\item itemize 
	\item Umgebung
	\item in LaTeX
\end{compactenum}

\begin{description}
\item[Apfel] ein Obst
\item[Birne] auch ein Obst
\item[Tomate] eher Gemüse
\end{description}

\begin{compactdesc}
\item[Apfel] ein Obst
\item[Birne] auch ein Obst
\item[Tomate] eher Gemüse
\end{compactdesc}


\blindtext

\blindtext

\subsubsection{Literatur nach 1900}

\blindtext

\blindtext

\blindtext

\section{Hauptteil}\label{sec:hauptteil}

\blindtext[100]


\section{Schlussworte}

\blindtext[10]

\begin{tabular}{|l|r|c|p{70mm}|} \hline
\textbf{1. Spalte} & \textbf{2. Spalte} & \textbf{3. Spalte} & \textbf{4. Spalte} \\ \hline \hline
Hallo Welt & 1.2345 & 123 & Hallo, ich bin ein kleiner Text. \\ \hline
Hallo Welt & 1.2345 & 123 & Hallo, ich bin ein kleiner Text. \\ \hline
Welt & 1.23 & 465465465 & Hallo \\ \hline
Hallo Welt & 1.2345 & 123 & Hallo, ich bin ein kleiner Text. \\ \hline
Hallo Welt & 1.2345 & 123 & Hallo, ich bin ein kleiner Text. \\ \hline
\end{tabular}\vspace*{2cm}

\begin{center}
\begin{tabular}{lrrp{60mm}} \toprule[2pt]
\textbf{1. Spalte} & \textbf{2. Spalte} & \textbf{3. Spalte} & \textbf{4. Spalte} \\ \cmidrule[1pt](rl){1-4}
Hallo Welt & 1.2345 & 123 & Hallo, ich bin ein kleiner Text. \\ 
Hallo Welt & 1.2345 & 123 & Hallo, ich bin ein kleiner Text. \\ 
Welt & 1.2300 & 465465465 & Hallo \\ 
Hallo Welt & 1.2345 & 123 & Hallo, ich bin ein kleiner Text. \\ 
Hallo Welt & 1.2345 & 123 & Hallo, ich bin ein kleiner Text. \\ \bottomrule
\end{tabular}
\end{center}

\begin{center}
\Large Hallo DLR!
\end{center}

\end{document}

