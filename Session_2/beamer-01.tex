\documentclass[ngerman]{beamer}
\usepackage{babel}
%\usepackage[version=4]{mhchem}
\usepackage{chemformula}

\author{Uwe Ziegenhagen}
\title{Meine erste Präsentation \ch{CO2} \\}
\institute{Köln}
\usetheme{AnnArbor}

\begin{document}

\begin{frame}
\maketitle
\end{frame}

\begin{frame}
\tableofcontents
\end{frame}


\section{About me}

\frame{
\frametitle{Einführung}
\framesubtitle{Literaturüberblick}

\begin{itemize}
	\item \ch{3 H2O} \\
	\item b
	\item c
\end{itemize}

\begin{equation}
-\frac{p}{2} \pm \sqrt{\left(\frac{p}{2} \right)^2 -q}
\end{equation}

}

\begin{frame}
\frametitle{Miezekatze in da house}

\begin{center}
\includegraphics[width=0.8\textwidth]{Bilder/Katze2}
\end{center}

\end{frame}

\section{Findings}
\begin{frame}
\frametitle{Miezekatze in da house}

\begin{columns}
\begin{column}{0.49\textwidth}
\begin{itemize}
	\item 
	\item 
	\item 
	\item 
	\item 
	\item 
\end{itemize}
\end{column}
\begin{column}{0.49\textwidth}
\includegraphics[width=\textwidth]{Bilder/Katze2}
\end{column}
\end{columns}

\end{frame}

\begin{frame}[allowframebreaks] % fragile bei Listings

\begin{itemize}
	\item 
	\item 
	\item 
	\item 
	\item 
	\item 
	\item 
	\item 
	\item 
	\item fsfs \newpage 
	\item 
	\item 
	\item 
	\item 
	\item 
	\item 
	\item 
	\item 
\end{itemize}

\end{frame}



\begin{frame}

\begin{itemize}
	\item Hallo \pause
	\item ich \pause
	\item bin  \pause
	\item eine \pause 
	\item Bullet
	\item Liste
\end{itemize}

\end{frame}

\section{Aufdeckungen}

\begin{frame}

\begin{itemize}
	\item<1-> Hallo 
	\item<2> ich 
	\item<-2> bin  
	\item<3-> eine 
	\item<1> Bullet
	\item<2> Liste
\end{itemize}

\end{frame}


\begin{frame}

\begin{itemize}
	\item<1-> Hallo 
	\item<2-> ich 
	\item<3-> bin  
	\item<4-> eine 
	\item<5-> Bullet
	\item<6-> Liste
\end{itemize}

\end{frame}

\end{document}